\documentclass{beamer}
\usepackage[spanish]{babel}
\usepackage[utf8]{inputenc}
\usepackage{lmodern} % para cualquier tamaño de letra
\usetheme{Copenhagen}
\setbeamercovered{transparent=30}

\begin{document}

\title{Compresión JPEG}
\author{Balbi Pablo Luis, Pazos Méndez Nicolás Javier}
\date{\today}

\begin{frame}
    \titlepage
\end{frame}

\begin{frame}
    \frametitle{Cómo viene la cosa...}
    \tableofcontents[hideallsubsections]
\end{frame}

\section{El método de compresión JPEG}
\begin{frame}
    \frametitle{El método de compresión JPEG}
        Cuatro pasos fundamentales:
        \begin{itemize}
            \item Separar la imagen en cuadrados de $8 \times 8$
            \item Aplicar la DCT\footnote{transformada del coseno discreta} a cada cuadrado
            \item Cuantizar los coeficientes del paso anterior
            \item Comprimir el conjunto de coeficientes finales
        \end{itemize}

\end{frame}

\end{document}
