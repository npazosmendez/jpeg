\documentclass{beamer}
\usepackage[spanish]{babel}
\usepackage[utf8]{inputenc}
\usepackage{lmodern} % para cualquier tamaño de letra
\usetheme{Copenhagen}
\setbeamercovered{transparent=30}

\begin{document}

\title{Compresión JPEG}
\author{Balbi Pablo Luis, Pazos Méndez Nicolás Javier}
\date{\today}

\begin{frame}
    \titlepage
\end{frame}

\begin{frame}
    \frametitle{Cómo viene la cosa...}
    \tableofcontents[hideallsubsections]
\end{frame}

\section{El método de compresión JPEG}
\begin{frame}
    \frametitle{El método de compresión JPEG}
        Cuatro pasos fundamentales:
        \begin{enumerate}
            \item Separar la imagen en cuadrados de $8 \times 8$
            \item Aplicar la DCT\footnote{transformada del coseno discreta} a cada cuadrado
            \item Cuantizar los coeficientes del paso anterior
            \item Comprimir el conjunto de coeficientes finales
        \end{enumerate}

\end{frame}

\begin{frame}
    \frametitle{Separar la imagen en cuadrados de $8 \times 8$}
    \begin{minipage}[t]{0.48\linewidth}
        \includegraphics[width=5cm, height=5cm]{fig/lena.png}
    \end{minipage}
    \hfill
    \begin{minipage}[t]{0.48\linewidth}
        \includegraphics[width=5cm, height=5cm]{fig/lena_blocks.png}
    \end{minipage}
\end{frame}

\begin{frame}
    \frametitle{Separar la imagen en cuadrados de $8 \times 8$}
    \textbf{¿Por qué partir la imagen en bloques?}
    \begin{itemize}
        \item Aplicar la DCT a la imagen completa requiere más memoria y es menos modular. % sirve que sea modular para cosas tipo GPU
        \item Las imágenes no suelen ser completamente homogéneas. Analizar las frecuencias de la imagen completa no es lo mejor para comprimir.
    \end{itemize}
\end{frame}

\begin{frame}
    \frametitle{Separar la imagen en cuadrados de $8 \times 8$}
    \includegraphics[width=5cm, height=5cm]{fig/no_homogenea.png}
\end{frame}

\begin{frame}
    \frametitle{Aplicar la DCT a cada cuadrado}
    [TODO] shifter -128 y aplicar la transformada

    [TODO] por qué usamos este espacio? porque las frecuencias bajas nos importan más

    [TODO] darles nombres a los coeficientes DC/AC para referirse después
\end{frame}

\begin{frame}
    \frametitle{Cuantizar los coeficientes}
    [TODO] por qué cuantizamos? Para comprimir más y más fácil. bajamos la entropía.
    Es importante entender que no estamos descartando por frecuencias, sino
    que estamos estableciendo como un step.

    [TODO] cómo cuantizamos? Con la tabla de cuantización
\end{frame}

\begin{frame}
    \frametitle{Comprimir el conjunto de coeficientes finales}

    [TODO] ahora que bajamos mucho la entropía podemos comprimir mucho más. Pasos:
    \begin{enumerate}
        \item Codificar los coeficientes DC
        \item Desarmar en $zig-zag$ cada cuadrado
        \item Comprimir cada cuadrado como tuplas $(\# ceros \, antes, \, valor)$
        \item Codificar las tuplas con Huffman

    \end{enumerate}
\end{frame}

\end{document}
